\documentclass[preview,12pt]{article}
\usepackage{amsmath}
\usepackage{gensymb}
\usepackage{ragged2e}
\usepackage{geometry}
\usepackage{graphicx}
\usepackage{caption}
\usepackage{subcaption}
\usepackage{pdfpages}

\geometry{letterpaper, margin=1in}

\begin{document}

\noindent PDE and Fourier\newline
Josh Coffey \newline
Homework 2 \newline
2/11/2020 \newline

\section*{(1)}
    Use separation of variables to solve the follwing initial/boundary value problem:
    $$\frac{\partial u}{\partial t}-\frac{\partial^2 u}{\partial x^2}=0$$
    $$\textrm{BC: }\frac{\partial u}{\partial x}(0,t)=\frac{\partial u}{\partial x}(\pi,t)=0$$
    with the different inital conditions in (b) and (c)
    \subsection*{(a)}
        Provide a full description of your solution to the eigenfunction problem:
        $$X''+\lambda X=0 \textrm{ with BCs } X'(0)=X'(\pi)=0$$
        Begin by guessing that a solution for the above PDE is:
        $$u(x,t)=X(x)T(t)$$
        $$\frac{\partial u}{\partial t}=X(x)T'(t)$$
        $$\frac{\partial^2u}{\partial x^2}=X''(x)T(t)$$
        Plugging these into the PDE gives: 
        $$X(x)T'(t)-X''(x)T(t)=0$$
        Rearranging and dividing by $X(x)T(t)$ gives:
        $$\frac{X''}{X}=\frac{T'}{T}$$
        Variables are now separate, but both sides of equation must be constant because changing one variable would result in an equivalent change in the other.  Can name that constant $-\lambda$ so that.
        $$\frac{X''}{X}=\frac{T'}{T}=-\lambda$$
        Creating two separate equations results in:
        $$X''=-\lambda X$$
        $$T'=-\lambda T$$
        Using the boundary conditions:
        $$X(0)=0$$
        $$X(\pi)=0$$
        $$X'(0)=0$$
        $$X'(\pi)=0$$
        Let $\lambda=0$
        $$X''=0$$
        $$X'=C_1$$
        $$X=C_1x+C_2$$
        $$X(0)=X(\pi)=0=C_1(0)+C_2=C_1(\pi)+C_2 \implies C_2=0$$
        $$X'(0)=X'(\pi)=C_1=0$$
        This is not useful information \newline
        Let $\lambda>0$ (positive), which gives
        $$X''+\lambda X=0$$
        Guess that 
        $$X=C_1cos(\sqrt{\lambda}x)+C_2sin(\sqrt{\lambda}x)$$
        $$X'=-\sqrt{\lambda}C_1sin(\sqrt{\lambda}x)+C_2\sqrt{\lambda}cos(\sqrt{\lambda}x)$$
        Using boundary conditions:
        $$X'(0)=0=-\sqrt{\lambda}C_1sin(\sqrt{\lambda}(0))+C_2\sqrt{\lambda}cos(\sqrt{\lambda}(0)) \implies C_2=0$$
        $$X'(\pi)=0=-\sqrt{\lambda}C_1sin(\sqrt{\lambda}(\pi)) \implies \sqrt{\lambda}\pi=n\pi$$
        $$n\in \mathbb{Z} \implies \lambda_n=n^2$$
        $$X_n(x)=cos(n x)$$
        Let $\lambda<0$ and guess that 
        $$X=C_1cosh(\mu x)+C_2sinh(\mu x)$$
        where $\lambda=-\mu^2$ \newline
        using boundary conditions:
        $$0=u_x(0.t)=u_x(\pi,t)=X'(0)=X'(\pi)$$
        $$X'(x)=\mu C_1 sinh(\mu x)+\mu C_2 cosh (\mu x)$$
        $$X'(0)=0+C_2 \mu (1) \implies C_2=0$$
        $$X'(\pi)=\mu C_1 sinh(\mu \pi)$$
        $$sinh(\mu\pi)\neq 0 \implies C_1=0$$
        This means that there no new eigenfunctions or eigenvalues other than $X_n(x)$. \newline
        Using the $T'=-\lambda T$ equation and the above value of $\lambda$: 
        $$T_n'=n^2T_n$$
        Solving this ODE gives:
        $$T_n=A_ne^{-n^2t}$$
        Combining these two equations gives the solution:
        $$u_n(x,t)=A_ne^{-n^2t}cos(n x); n=0,1,2,...$$
        
    \subsection*{(b)}
        Using the result from (a) and the given IC: $u(x,0)=f(x)=1+3cos(5x)$ \newline
        Plugging "0" in for $t$ gives:
        $$u(x,0)=A_ncos(nx)$$
        By inspection and the principle of superposition, choose n=0 and n=5.  This implies that $A_0=1$, $A_5=3$, and all other $A_n=0$. \newline
        This gives the solution:
        $$u(x,t)=e^{-t}cos(x)+3e^{-t}cos(x)$$
        
    \subsection*{(c)}
        Using the result from (a), the given IC: $u(x,0)=f(x)=x^2$, and the fact that 
        $$x^2\sim \frac{\pi^2}{3}+4\Sigma^\infty_{n=1}\frac{(-1)^ncos(nx)}{n^2}$$
        $$u(x,0)=A_ncos(nx)=x^2 \sim \frac{\pi^2}{3}+4\Sigma^\infty_{n=1}\frac{(-1)^ncos(nx)}{n^2}$$
        Using inspection, when n=0, $A_0=\frac{\pi^2}{3}$ and when n$\neq$0, $A_n=\frac{4(-1)^n}{n^2}$, so that
        $$u(x,t)=\frac{\pi^2}{3}e^{-t}cos(x)+e^{-t}\left(4\Sigma^\infty _{n=1}\frac{(-1)^ncos(nx)}{n^2}\right)cos(x)$$
        
\section*{(2)}
    Find the separation of variables solution u=u(x,y) to the Laplace equation on rectangle R where $0<x<\pi$ and $0<y<1$;
    $$\frac{\partial^2 u}{\partial x^2}+\frac{\partial^2u}{\partial y^2}=0 \textrm{ with BC } u(x,0)=0 \textrm{ \& } u(x,1)=sin(2x) \textrm{ and } u(0,y)=0 \textrm{ \& } u(\pi,y)=0$$
    (you may use our results for the problem $X''+\lambda X=0$ with $X(0)=0$ \& $X(\pi)=0$ from lecture).
    \newline
    From superposition know that:
    $$u=u_1+u_2+u_3+u_4=0+0+sin(2x)+0$$
    $$\nabla^2 u=u_{xx}+u_{yy}=0 \textrm{ in R}$$
    Using boundary conditions:
    $$u(x,1)=sin(2x)$$
    Guess that 
    $$u(x,y)=X(x)Y(y)\implies-\frac{Y''}{Y}=\frac{X''}{X}=-\lambda$$
    This gives
    $$X''+\lambda X=0$$
    and 
    $$Y''-\lambda Y=0$$
    boundary conditions:
    $$Y(0)=0$$
    $$X(0)=0$$
    $$X(\pi)=0$$
    Using previous results:
    $$X_n(x)=sin(\frac{n\pi x}{\pi})$$
    With 1 replacing H and $\pi$ replacing L. \newline
    Using previously solved Y solutions:
    $$Y_n(y)=A_nsinh(n(y-1))$$
    Thus: 
    $$u_n(x,y)=A_nsin(nx)sinh(n(y-1))$$
    and 
    $$u(x,y)=\Sigma^\infty_{n=1} A_n u_n(x,y)$$
    Know the general solution for $A_n$
    $$A_n=-\frac{2}{Lsinh(\frac{n\pi H}{L})}\int_0^Lg(x)sin(\frac{n\pi x}{L})dx$$
    Replacing H and L with 1 and $\pi$ and g(x) with sin(2x):
    $$A_n=-\frac{2}{\pi sinh(\frac{n\pi 1}{\pi})}\int_0^\pi sin(2x)sin(\frac{n\pi x}{\pi})dx$$
    $$A_n=-\frac{2}{\pi sinh(n)}\int_0^\pi sin(2x)sin(nx)dx$$
    This gives the solution for u(x,y)
    $$u(x,y)=-\Sigma^\infty_{n=1}\left(\frac{2}{\pi sinh(n)}\right)(sin(nx))(sinh(n(y-1)))\left[\int_0^\pi sin(2x)sin(nx)dx\right]$$
    
\section*{(3)}
    Determine the steady state temperature distribution for the thin circular ring (Heat equation on $-L\leq x \leq L$ with periodic BCs and no source $Q\equiv0$);
    \subsection*{(a)}
        Directly from the steady state equation (You will need to use the time-dependent partial differential equation problem as well) \newline
        Steady State heat equation:
        $$u_t=ku_{xx}$$
        $$k=\frac{k_0}{c\rho}$$
        Periodic boundary conditions means that
        $$u(L,t)=u(-L,t)$$
        $$u_x(L,t)=u_x(-L,t)$$
        Let
        $$u(x,t)=X(x)T(t)$$
        This leads to the two equations
        $$X''+\lambda X = 0$$
        $$T' = -\lambda k T \implies T=Ae^{-\lambda kt}$$
        Solving the X equation requires looking at the three cases of $\lambda$ 
        $$$$
        If $\lambda>0$
        $$X=C_1 cos(\sqrt{\lambda}x)+C_2sin(\sqrt{\lambda}x)$$
        Using the BCs for X:
        $$X(-L)=X(L) \implies 2C_2sin(\sqrt{\lambda}L)=0$$
        $$X'(-L)=X'(L) \implies 2C_1sin(\sqrt{\lambda}L)=0$$
        This implies that $\lambda_n=\frac{\n^2\pi^2}{L^2}$, n=1,2,3,... \newline
        Using this information to create X(x):
        $$X_n(x)=C_1cos(\frac{n\pi x}{L})+C_2sin(\frac{n\pi x}{L})$$
        $$$$
        If $\lambda=0$
        $$X''=0$$
        $$X(x)=C_1+C_2x$$
        Using boundary conditions:
        $$C_1-C_2L=C_1+C_2L \implies C_2=0$$
        $$0=0$$
        $$X_0=1, \lambda_0=0$$
        $$$$
        If $\lambda<0$, let $\lambda=-\mu^2$ where $\mu>0$
        Guess that 
        $$X(x)=C_1cosh(\mu x)+C_2 sinh(\mu x)$$
        Using BCs and fact that cosh is even and sinh is odd:
        $$X(L)=X(-L) \implies C_2sinh(\mu L)=0$$
        Because sinh($\mu$L)$\neq$0, $C_2=0$. \newline
        Using X'(L)=X'(-L)
        $$C_1\mu sinh(\mu(-L))=C_1\mu sinh(\mu(L))$$
        $$2C_1\mu sinh(\mu L)=0$$
        Either $C_1, \mu,$ or $sinh(\mu L)$ is 0, but only $C_1$ is able to be zero. \newline
        Using these equations together gives:
        $$u(x,t)=A_0+\Sigma_{n=1}^\infty A_ne^{-\frac{n^2\pi^2kt}{L^2}}cos\frac{n\pi x}{L}+\Sigma_{n=1}^\infty B_ne^{-\frac{n^2\pi^2kt}{L^2}}sin\frac{n\pi x}{L}$$
        
    \subsection*{(b)}
        Computing the limit as $t\rightarrow{}\infty$ of the previous series: \newline
        As $t\rightarrow{}\infty$, the exponentials will go to zero.  This will leave 
        $$u(x,\infty)=A_0$$
        From class, know that
        $$A_0=\frac{1}{2L}\int_{-L}^Lf(x)dx$$
        and thus depends on the initial conditions of the circular ring.

\section*{(4)}
    For Laplace's equation inside a circular disk ($r<a$) with BC $u(a,\theta)=f(\theta)$, it is known that
    $$u(r,\theta)=\Sigma_{n=0}^\infty A_nr^ncos(n\theta)+\Sigma_{n=1}^\infty B_nr^nsin(n\theta)$$
    with (for n=1,2,...)
    $$A_0=\frac{1}{2\pi}\int_{-\pi}^\pi f(\phi)d\phi,\textrm{ } A_na^n=\frac{1}{\pi}\int_{-\pi}^\pi f(\phi)cos(n\phi)d\phi \textrm{ and } B_na^n=\frac{1}{\pi}\int_{-\pi}^\pi f(\phi)sin(n\phi)d\phi$$
    \subsection*{(a)}
        Show that
        $$u(r,\theta)=\frac{1}{\pi}\int_{-\pi}^\pi f(\phi)\left[-\frac{1}{2}+\Sigma_{n=0}^\infty \left(\frac{r}{a}\right)^n cos(n(\theta-\phi))\right]d\phi$$
        Begin by noting that:
        $$A_0+\Sigma_{n=1}^\infty A_nr^ncos(n\theta)=\Sigma_{n=0}^\infty A_nr^ncos(n\theta)$$
        $$u(r,\theta)=A_0+\Sigma_{n=1}^\infty A_nr^ncos(n\theta)+\Sigma_{n=1}^\infty B_nr^nsin(n\theta)=A_0+\Sigma_{n=1}^\infty A_nr^ncos(n\theta)+B_nr^nsin(n\theta)$$
        Now, plug in values for $A_n$ and $B_n$, with the $a^n$ divided to the other side
        $$u(r,\theta)=A_0+\Sigma_{n=1}^\infty \frac{1}{\pi}\int_{-\pi}^\pi f(\phi)cos(n\phi)d\phi r^ncos(n\theta)\frac{1}{a^n}+\frac{1}{\pi}\int_{-\pi}^\pi f(\phi)sin(n\phi)d\phi r^nsin(n\theta)\frac{1}{a^n}$$
        $$u(r,\theta)=A_0+\Sigma_{n=1}^\infty \frac{1}{\pi}\int_{-\pi}^\pi f(\phi)cos(n\phi)d\phi cos(n\theta)\frac{r^n}{a^n}+\frac{1}{\pi}\int_{-\pi}^\pi f(\phi)sin(n\phi)d\phi sin(n\theta)\frac{r^n}{a^n}$$
        $$u(r,\theta)=\frac{1}{2\pi}\int_{-\pi}^\pi f(\phi)d\phi+\frac{1}{\pi}\Sigma_{n=1}^\infty \left(\frac{r}{a}\right)^n\left[\int_{-\pi}^\pi f(\phi)cos(n\phi)d\phi cos(n\theta)+\int_{-\pi}^\pi f(\phi)sin(n\phi)d\phi sin(n\theta)\right]$$
        $$u(r,\theta)=\frac{1}{\pi}\left[\frac{1}{2}\int_{-\pi}^\pi f(\phi)d\phi+\Sigma_{n=1}^\infty \left(\frac{r}{a}\right)^n\left[\int_{-\pi}^\pi f(\phi)cos(n\phi)d\phi cos(n\theta)+\int_{-\pi}^\pi f(\phi)sin(n\phi)d\phi sin(n\theta)\right]\right]$$
        Can put the $\theta$ cosine and sine inside the integral since constant with respect to $\phi$
        $$u(r,\theta)=\frac{1}{\pi}\left[\frac{1}{2}\int_{-\pi}^\pi f(\phi)d\phi+\Sigma_{n=1}^\infty \left(\frac{r}{a}\right)^n\left[\int_{-\pi}^\pi f(\phi)cos(n\phi)cos(n\theta)d\phi+\int_{-\pi}^\pi f(\phi)sin(n\phi)sin(n\theta)d\phi \right]\right]$$
        Simplifying gives:
        $$u(r,\theta)=\frac{1}{\pi}\left[\int_{-\pi}^\pi f(\phi)\left[\frac{1}{2}+\Sigma_{n=1}^\infty \left(\frac{r}{a}\right)^n\left[cos(n\phi)cos(n\theta)+sin(n\phi)sin(n\theta) \right]\right]\right]d\phi$$
        Using trig identity:
        $$cos(\alpha \pm \beta)=cos\alpha cos\beta \mp sin\alpha sin\beta$$
        $$u(r,\theta)=\frac{1}{\pi}\left[\int_{-\pi}^\pi f(\phi)\left[\frac{1}{2}+\Sigma_{n=1}^\infty \left(\frac{r}{a}\right)^n\left[cos(n\theta - n\phi) \right]\right]\right]d\phi$$
        $$u(r,\theta)=\frac{1}{\pi}\left[\int_{-\pi}^\pi f(\phi)\left[\frac{1}{2}+\Sigma_{n=1}^\infty \left(\frac{r}{a}\right)^n\left[cos(n(\theta-\phi)) \right]\right]\right]d\phi$$
        Changing the beginning index by subtracting $A_0$ gives
        $$u(r,\theta)=\frac{1}{\pi}\left[\int_{-\pi}^\pi f(\phi)\left[-\frac{1}{2}+\Sigma_{n=0}^\infty \left(\frac{r}{a}\right)^n\left[cos(n(\theta-\phi)) \right]\right]\right]d\phi$$
        
    \subsection*{(b)}
        Show, using cos$z=Re(e^{iz})$ and summing the resulting gemoetric series, \textit{Poisson's Integral Formula}:
        $$u(r,\theta)=\frac{a^2-r^2}{2\pi}\int_{-\pi}^\pi\frac{f(\phi)d\phi}{a^2+r^2-2racos(\theta-\phi)}$$
        Know that
        $$e^{iz}=cosz+isinz$$
        Using previous result and knowing that $\Sigma_{n=0}^\infty \left(\frac{r}{a}\right)^ncos(n\psi)=Re\left[\Sigma_{n=0}^\infty \left(\frac{r}{a}\right)^ne^{in\psi}\right]$, where $\psi=\theta-\phi$
        $$u(r,\theta)=\frac{1}{\pi}\left[\int_{-\pi}^\pi f(\phi)\left[-\frac{1}{2}+\Sigma_{n=0}^\infty Re\left[\Sigma_{n=0}^\infty \left(\frac{r}{a}\right)^ne^{in(\theta-\phi)}\right]\right]\right]d\phi$$
        Know that $\abs{\frac{r}{a}}<1$, so can use fact that $\Sigma_{n=0}^\infty C^n=\frac{1}{1-C}$
        $$u(r,\theta)=\frac{1}{\pi}\left[\int_{-\pi}^\pi f(\phi)\left[-\frac{1}{2}+\Sigma_{n=0}^\infty \left(\frac{1}{1-\frac{r}{a}}\right)cos(n(\theta-\phi))\right]\right]d\phi$$
        Note that 
        $$\Sigma_{n=0}^\infty cos(n(\theta-\phi)) = cos(\theta-\phi)$$
        
\end{document}